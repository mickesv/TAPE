\documentclass[10pt]{beamer}

%% Use this for 4 on 1 handouts
%\documentclass[handout]{beamer}
%\usepackage{pgfpages}
%\pgfpagesuselayout{4 on 1}[landscape, a4paper, border shrink=5mm]

\usepackage[english]{babel}
\usepackage[utf8]{inputenc}
\usepackage[T1]{fontenc}
\def\subitem{\item[\hspace{1.5cm} -]}
\usepackage{graphvizzz}

% Set the presentation mode to BTH
\mode<presentation>
{
	\usetheme{BTH_msv}
	% Comment this if you do not want to reveal the bullets before they are going to be used
	\setbeamercovered{transparent}
}


% Information for the title page

\title[]{APE ToolSuite}
\subtitle{A brief presentation}
%\date[]{}

\author[Mikael Svahnberg]{Mikael Svahnberg\inst{1}}
\institute[BTH] % (optional, but mostly needed)
{
  \inst{1}%
 Mikael.Svahnberg@bth.se\\
 School of Computing\\
 Blekinge Institute of Technology%
}

% Delete this, if you do not want the table of contents to pop up at
% the beginning of each subsection:
%\AtBeginSubsection[]
%{
%  \begin{frame}<beamer>{Outline}
%    \tableofcontents[currentsection,currentsubsection]
%  \end{frame}
%}


% If you wish to uncover everything in a step-wise fashion, uncomment
% the following command: 
%\beamerdefaultoverlayspecification{<+->}

\begin{document}

% Titlepage frame
\begin{frame}
  \titlepage
\end{frame}

% ToC frame
% Use \section and \subsection commands to get things into the ToC.
%\begin{frame}
 %\frametitle{Outline}
 % \tableofcontents
%\end{frame}

% -----------------------------
% Your frames goes here
% -----------------------------

\begin{frame}[t]
\frametitle{Overall structure}

\digraph[width=10cm]{FTopLevel}{
rankdir=LR;
e [label="Extract"];
p [label="Parse*"];
a [label="Analyse"];
v [label="Visualise"];
e->p->a->v;
}

\end{frame}

\begin{frame}[t]
\frametitle{Extract}

Extract:
\begin{itemize}
\item Functions
\item Global Variables
\item Classes
\item Class Inheritance
\item Methods
\item Attributes
\item (Files, Directories)
\item \ldots
\end{itemize}

Establishes the \emph{static structure}; the one intended by the programmers.

\begin{exampleblock}{Example}
\begin{scriptsize}
ProjectHasId;-1;1;name=this\\
ProjectHasFile;1;3;basePath=/Users/msv/Documents/Research/code\_analysis/,name=model.hh\\
FileDeclaresClass;3;11;name=ModelNode\\
ClassContainsAttribute;11;14;class=ModelNode,name=target\\
ClassContainsMethod;11;16;class=ModelNode,name=setId\\
\end{scriptsize}
\end{exampleblock}

\end{frame}

\begin{frame}[t]
\frametitle{Parse}

For each Function and Method, go through the motions of the compiler and apply a set of parsers:
\begin{itemize}
\item Function Call Parser -- Generates a call graph
\item Loop Depth Parser -- What is the maximum loop depth in which function A calls function B?
\item Function Size Parser -- What is the McCabe size of a function?
\item Global Variable Access Parser -- tracks usage of global variables
\item (Attribute Access Parser) -- tracks usage of class attributes
\item \ldots
\end{itemize}

Establishes the \emph{actual structure}; the structure as implemented.

\begin{exampleblock}{Example}
\begin{scriptsize}
FunctionCallsFunction;178;173;count=4,maxLoopDepth=1,target=print\\
FunctionAccessVariable;154;202;count=3,target=aGlobalVariable\\
ClassContainsMethod;118;195;class=GVAccessParser,name=parse,size=23\\
\end{scriptsize}
\end{exampleblock}

\end{frame}

\begin{frame}[t]
\frametitle{Parse}
\begin{itemize}
\item Since we ``shadow'' the compiler, any type of parsing is possible.
\item The trick is to trigger on the right code elements, recurse into a statement, or continue to the next statement.
\item For example, in order to check access to global variables, you need to keep track of local variables; after this it is relatively easy to check their type (and size).
\item current examples include LoopDepthParser and FunctionSizeParser
\end{itemize}

% \begin{exampleblock}{Code Elements}
% \begin{scriptsize}
% CXCursorKind {
%   CXCursor_UnexposedDecl = 1, CXCursor_StructDecl = 2, CXCursor_UnionDecl = 3, CXCursor_ClassDecl = 4,
%   CXCursor_EnumDecl = 5, CXCursor_FieldDecl = 6, CXCursor_EnumConstantDecl = 7, CXCursor_FunctionDecl = 8,
%   CXCursor_VarDecl = 9, CXCursor_ParmDecl = 10, CXCursor_ObjCInterfaceDecl = 11, CXCursor_ObjCCategoryDecl = 12,
%   CXCursor_ObjCProtocolDecl = 13, CXCursor_ObjCPropertyDecl = 14, CXCursor_ObjCIvarDecl = 15, CXCursor_ObjCInstanceMethodDecl = 16,
%   CXCursor_ObjCClassMethodDecl = 17, CXCursor_ObjCImplementationDecl = 18, CXCursor_ObjCCategoryImplDecl = 19, CXCursor_TypedefDecl = 20,
%   CXCursor_CXXMethod = 21, CXCursor_Namespace = 22, CXCursor_LinkageSpec = 23, CXCursor_Constructor = 24,
%   CXCursor_Destructor = 25, CXCursor_ConversionFunction = 26, CXCursor_TemplateTypeParameter = 27, CXCursor_NonTypeTemplateParameter = 28,
%   CXCursor_TemplateTemplateParameter = 29, CXCursor_FunctionTemplate = 30, CXCursor_ClassTemplate = 31, CXCursor_ClassTemplatePartialSpecialization = 32,
%   CXCursor_NamespaceAlias = 33, CXCursor_UsingDirective = 34, CXCursor_UsingDeclaration = 35, CXCursor_TypeAliasDecl = 36,
%   CXCursor_ObjCSynthesizeDecl = 37, CXCursor_ObjCDynamicDecl = 38, CXCursor_CXXAccessSpecifier = 39, CXCursor_FirstDecl = CXCursor_UnexposedDecl,
%   CXCursor_LastDecl = CXCursor_CXXAccessSpecifier, CXCursor_FirstRef = 40, CXCursor_ObjCSuperClassRef = 40, CXCursor_ObjCProtocolRef = 41,
%   CXCursor_ObjCClassRef = 42, CXCursor_TypeRef = 43, CXCursor_CXXBaseSpecifier = 44, CXCursor_TemplateRef = 45,
%   CXCursor_NamespaceRef = 46, CXCursor_MemberRef = 47, CXCursor_LabelRef = 48, CXCursor_OverloadedDeclRef = 49,
%   CXCursor_VariableRef = 50, CXCursor_LastRef = CXCursor_VariableRef, CXCursor_FirstInvalid = 70, CXCursor_InvalidFile = 70,
%   CXCursor_NoDeclFound = 71, CXCursor_NotImplemented = 72, CXCursor_InvalidCode = 73, CXCursor_LastInvalid = CXCursor_InvalidCode,
%   CXCursor_FirstExpr = 100, CXCursor_UnexposedExpr = 100, CXCursor_DeclRefExpr = 101, CXCursor_MemberRefExpr = 102,
%   CXCursor_CallExpr = 103, CXCursor_ObjCMessageExpr = 104, CXCursor_BlockExpr = 105, CXCursor_IntegerLiteral = 106,
%   CXCursor_FloatingLiteral = 107, CXCursor_ImaginaryLiteral = 108, CXCursor_StringLiteral = 109, CXCursor_CharacterLiteral = 110,
%   CXCursor_ParenExpr = 111, CXCursor_UnaryOperator = 112, CXCursor_ArraySubscriptExpr = 113, CXCursor_BinaryOperator = 114,
%   CXCursor_CompoundAssignOperator = 115, CXCursor_ConditionalOperator = 116, CXCursor_CStyleCastExpr = 117, CXCursor_CompoundLiteralExpr = 118,
%   CXCursor_InitListExpr = 119, CXCursor_AddrLabelExpr = 120, CXCursor_StmtExpr = 121, CXCursor_GenericSelectionExpr = 122,
%   CXCursor_GNUNullExpr = 123, CXCursor_CXXStaticCastExpr = 124, CXCursor_CXXDynamicCastExpr = 125, CXCursor_CXXReinterpretCastExpr = 126,
%   CXCursor_CXXConstCastExpr = 127, CXCursor_CXXFunctionalCastExpr = 128, CXCursor_CXXTypeidExpr = 129, CXCursor_CXXBoolLiteralExpr = 130,
%   CXCursor_CXXNullPtrLiteralExpr = 131, CXCursor_CXXThisExpr = 132, CXCursor_CXXThrowExpr = 133, CXCursor_CXXNewExpr = 134,
%   CXCursor_CXXDeleteExpr = 135, CXCursor_UnaryExpr = 136, CXCursor_ObjCStringLiteral = 137, CXCursor_ObjCEncodeExpr = 138,
%   CXCursor_ObjCSelectorExpr = 139, CXCursor_ObjCProtocolExpr = 140, CXCursor_ObjCBridgedCastExpr = 141, CXCursor_PackExpansionExpr = 142,
%   CXCursor_SizeOfPackExpr = 143, CXCursor_LambdaExpr = 144, CXCursor_ObjCBoolLiteralExpr = 145, CXCursor_LastExpr = CXCursor_ObjCBoolLiteralExpr,
%   CXCursor_FirstStmt = 200, CXCursor_UnexposedStmt = 200, CXCursor_LabelStmt = 201, CXCursor_CompoundStmt = 202,
%   CXCursor_CaseStmt = 203, CXCursor_DefaultStmt = 204, CXCursor_IfStmt = 205, CXCursor_SwitchStmt = 206,
%   CXCursor_WhileStmt = 207, CXCursor_DoStmt = 208, CXCursor_ForStmt = 209, CXCursor_GotoStmt = 210,
%   CXCursor_IndirectGotoStmt = 211, CXCursor_ContinueStmt = 212, CXCursor_BreakStmt = 213, CXCursor_ReturnStmt = 214,
%   CXCursor_GCCAsmStmt = 215, CXCursor_AsmStmt = CXCursor_GCCAsmStmt, CXCursor_ObjCAtTryStmt = 216, CXCursor_ObjCAtCatchStmt = 217,
%   CXCursor_ObjCAtFinallyStmt = 218, CXCursor_ObjCAtThrowStmt = 219, CXCursor_ObjCAtSynchronizedStmt = 220, CXCursor_ObjCAutoreleasePoolStmt = 221,
%   CXCursor_ObjCForCollectionStmt = 222, CXCursor_CXXCatchStmt = 223, CXCursor_CXXTryStmt = 224, CXCursor_CXXForRangeStmt = 225,
%   CXCursor_SEHTryStmt = 226, CXCursor_SEHExceptStmt = 227, CXCursor_SEHFinallyStmt = 228, CXCursor_MSAsmStmt = 229,
%   CXCursor_NullStmt = 230, CXCursor_DeclStmt = 231, CXCursor_LastStmt = CXCursor_DeclStmt, CXCursor_TranslationUnit = 300,
%   CXCursor_FirstAttr = 400, CXCursor_UnexposedAttr = 400, CXCursor_IBActionAttr = 401, CXCursor_IBOutletAttr = 402,
%   CXCursor_IBOutletCollectionAttr = 403, CXCursor_CXXFinalAttr = 404, CXCursor_CXXOverrideAttr = 405, CXCursor_AnnotateAttr = 406,
%   CXCursor_AsmLabelAttr = 407, CXCursor_LastAttr = CXCursor_AsmLabelAttr, CXCursor_PreprocessingDirective = 500, CXCursor_MacroDefinition = 501,
%   CXCursor_MacroExpansion = 502, CXCursor_MacroInstantiation = CXCursor_MacroExpansion, CXCursor_InclusionDirective = 503, CXCursor_FirstPreprocessing = CXCursor_PreprocessingDirective,
%   CXCursor_LastPreprocessing = CXCursor_InclusionDirective, CXCursor_ModuleImportDecl = 600, CXCursor_FirstExtraDecl = CXCursor_ModuleImportDecl, CXCursor_LastExtraDecl = CXCursor_ModuleImportDecl
% }
% \end{scriptsize}
% \end{exampleblock}

\end{frame}

\begin{frame}[t]
\frametitle{Analyse}
\begin{itemize}
\item Separate program that operates on the csv-data.
\item Originally intended to analyse for components, but has evolved as we learnt more.
\item Development agenda:
\begin{itemize}
\item Create \emph{actual} class structure (ignoring the static structure)
\item Analyse for e.g. performance patterns on design pattern level.
\item Create meaningful components and a meaningful architecture representation
\item Analyse for e.g. performance patterns on architecture level (e.g. correlate call structure with execution architecture)
\end{itemize}
\end{itemize}
\end{frame}

\begin{frame}[t]
\frametitle{Visualise}

\begin{itemize}
\item Not the primary focus right now.
\item Use GraphViz to produce \emph{something}.
\end{itemize}

\begin{exampleblock}{Example}
\begin{scriptsize}
// Function Nodes\\
29 [label="<<Function>> addBasics"];

// Method Call Arches\\
103->126 [label="<<M\_Call>>"];\\
\end{scriptsize}
\end{exampleblock}
\end{frame}

\begin{frame}[t]
\frametitle{Visualize}

\href{run:thisFull.dot}{\includegraphics[width=12cm]{thisFull.pdf}}
\end{frame}





% -----------------------------


%% All of the following is optional and typically not needed. 
%\appendix
%\begin{frame}[allowframebreaks]
%  \frametitle{For Further Reading}
%    
%  \begin{thebibliography}{10}
%    
%  \beamertemplatebookbibitems
%  % Start with overview books.

%  \bibitem{Author1990}
%    A.~Author.
%    \newblock {\em Handbook of Everything}.
%    \newblock Some Press, 1990.
%     
%  \beamertemplatearticlebibitems
%  % Followed by interesting articles. Keep the list short. 

%  \bibitem{Someone2000}
%    S.~Someone.
%    \newblock On this and that.
%    \newblock {\em Journal of This and That}, 2(1):50--100,
%    2000.
%  \end{thebibliography}
%\end{frame}

\end{document}


